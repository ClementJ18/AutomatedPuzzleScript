\chapter{Discussion}
\label{ch:discussion}
In this chapter, we discuss the results of our project.

\section{Design Formalization}
Our first contribution is a document outlining the syntax of a PuzzleScript file. We studied PuzzleScript's design by observing how the games behaved under various inputs and by going through the codebase. Based on this we created a document containing details on the design of PuzzleScript and the conclusion of our observations. This document is Chapter \ref{ch:puzzlescript} of the thesis. The aim of the document was to create documentation on the syntax structure of PuzzleScript so that we could parse it using a formal grammar. The document is complete enough for others to create their own parser and documents the corner cases of PuzzleScript. It lays out the basic constraints that certain objects must abide by. For all the constraints of PuzzleScript, refer to Appendix A.

\section{ScriptButler}
We then used our formalization of PuzzleScript's design to create our own design of an implementation which we then implemented. Our goal was to create a more extensible and maintainable implementation of PuzzleScript. This prototype is not a feature-complete implementation of PuzzleScript as can be seen in Table \ref{fig:feature_comparison}. However, it is complete enough to provide feedback on a wide range of games. This range includes published games.

\begin{figure}[!t]
    \centering
    \caption{Feature Comparison between the original implementation and our implementation.}
\begin{tabular}{l|l|l|l}
    \textbf{Category} & \textbf{Feature}           & \textbf{JavaScript} & \textbf{Rascal}       \\ \hline
    \textit{Editor}   & Text Editor       & Yes                       & Yes                         \\
             & Syntax Coloring   & Yes                       & Yes                         \\
             & Messages          & No                        & Yes                         \\
             & Game frame        & Yes                       & Yes                         \\
             & Autocomplete      & Yes                       & No                         \\
             & Export            & Yes                       & No                          \\
             & Share             & Yes                       & No                          \\
             & Save states       & Yes                       & No                          \\
             & Keyboard control  & Yes                       & Yes                         \\
             & Debug             & Yes                       & Yes, more extensive         \\
    \textit{Parser}   & Error Checking    & Yes                       & Yes, generic error          \\
             & Grammar           & Implicit                  & Formal                      \\
    \textit{Checker}  & Error Checking    & N/A                       & Yes                         \\
             & Errors            & 101                       & 72                          \\
             & Warnings          & 19                        & 11                          \\
             & Error Isolation   & No, cascades              & Yes                         \\
    \textit{Compiler} & Error Checking    & Yes                       & None                        \\
             & Prelude           & Implemented               & Implemented up to Checking  \\
             & Objects           & Implemented               & Implemented                 \\
             & Legend            & Implemented               & Implemented                 \\
             & Sounds            & Implemented               & Implemented up to Checking  \\
             & Layers            & Implemented               & Implemented                 \\
             & Rules             & Implemented               & Implemented, mostly working \\
             & Win Conditions    & Implemented               & Implemented                 \\
             & Levels            & Implemented               & Implemented                 \\
    \textit{Engine}   & Randomness        & Implemented               & Implemented                 \\
             & Rigid Bodies      & Implemented               & Not Implemented             \\
             & Real time         & Implemented               & Not Implemented             \\
             & Rule Loops        & Implemented               & Not Implemented             \\
    \end{tabular}
    \label{fig:feature_comparison}
\end{figure}


The grammar we created using our design document can parse most games, but it struggles with parsing comments. Currently, we strip the comments before parsing similarly to the JavaScript implementation. In the future, parsing comments could prove an interesting source for metrics. Ultimately, the grammar is a very literal representation of a PuzzleScript file that can be easily extended and modified to alter PuzzleScript's design. The static checker is fully functional but does not currently check for the errors of the original implementation. This is an intentional design decision, we instead focused on implementing new types of errors within our checker that the original implementation did not check for. There are no obstacles to reimplementing all the error messages but they did not provide a lot of value for this project. The compiler and engine work on a wide range of simple games. We do not currently implement complex features such as real-time or rigid bodies. These features are complex to implement and not commonly used in games. We design and implement the engine as a collection of plugins that can easily be exchanged. This design decision should make it easier to optimize in the future.

The final result is a partial reimplementation of the PuzzleScript engine. This implementation can run most simple games, is more extensible and more maintainable than the JavaScript one. We achieve this by reducing the coupling and code density. We leveraged Rascal features to make the compiled Rules more readable and verify the functionality using the extensive test suite discussed in \ref{sec:testing}.

\section{Supporting Game Designers}
To support game designers, we implemented three additional components into our tool: 1) an IDE, 2) an interface for running games and playtesting, and 3) a dynamic analyzer for automated gameplay testing.

The IDE allows game designers to edit their games, similarly to the in-browser editor of the original JavaScript implementation. It provides syntax coloring, an outline, a contextual menu and displays error messages. The IDE allows designers to access the feedback generated by ScriptButler in a user-friendly way as it directly attaches it next to the problematic line. We provide several other features that the original editor does not provide. However, we do not support the extra features that the original editor had such as the ability to share the game on GitHub or to export it as an HTML page. The reason is that does features do not have an impact on game quality. The only feature from the current editor we would potentially look at reimplementing is the 'Rebuild' feature. This feature re-compiles the game while it is running, allowing changes to game mechanics, victory conditions, and object appearance to take effect immediately. 

The interface implements the ability for designers to run and play their games. This component was separated from the IDE, unlike the original implementation, so that we could expand it. The interface provides the game designers with tools that can be used to debug their game. An important difference between our implementation and the original's is that our editor does not use keyboard buttons but rather requires the player to click on UI buttons. This error should be fixed in future works when salix supports keyboard events. Future works could also split the interface between the 'game designer view' and the 'tool developer view' to enhance the purposes of both.

Finally, the dynamic analyzer is similar to the static checker. It generates error messages that are then processed by the IDEs and displayed to game designers. However, the messages relate to the quality of the gameplay rather than the quality of the code. The dynamic analyzer provides the game designer with feedback on the playability of their game. In terms of playtesting, this feedback is shallow. However, we purposefully keep the feedback shallow so that we can provide it rapidly, it is an intentional design decision.


\section{Research Questions}
Here we discuss how our work contributes to answering our research questions. We begin by discussing the subquestions and then we describe how they help us answer our main research question.

%\textbf{RQ1.1}: How can PuzzleScript be used for studying game mechanics?
\subsection{RQ1.1} 
As we have discussed previously, PuzzleScript rules are very closely tied to the affordances they provide players. The match and replace patterns almost literally define game mechanics. This simplification allows us to more easily study the effect of code changes on game mechanics. We show this in the dynamic analyzer of ScriptButler. When the game designer removes a piece of code, the tool provides feedback on how it impacts the game mechanics required to solve certain levels or meet certain victory conditions. Despite being relatively simple, PuzzleScript is a powerful and popular game language. This popularity ensures that we study features that meet game designer's need. We could study a programming language, this would make the code easier to analyze. However, studying a programming language does not represent a realistic scenario. Game designers use game engines, so we have to study a game engine.

\subsection{RQ1.2} 
When iteratively designing, game developers evolve the software of their game over time, sometimes causing changes with a negative impact on the quality of gameplay. Game designers can reduce the impact of these changes by putting in place mitigation strategies such as playtesting. 

These strategies are time-consuming and unreliable. Playtesting requires a human to play through the entire game and actively attempt to break it. Sometimes, AI can be used to automatically play through the game and try to detect gameplay issues. However, this presents the same issue of how incredibly time-consuming this approach is, both in terms of time investment in putting together an AI that can play a game and having the AI test the game. Therefore, the game designers put in place tests to verify gameplay. These tests work similar to unit tests, they provide an input to the game and expect an output. This output is usually a specific game state. These tests allow for automated game design. However, the tests need to be adapted as the requirements change, creating additional dependencies. The tests can only be created within the context of a specific game and cannot be generally applied to a game engine.

\subsection{RQ1}
% \textbf{RQ1}: How can meta-programming tools and techniques be used to empower PuzzleScript game designers and allow them to freely explore the design space of their game?
We now know how PuzzleScript code relates to the code and why that relation is useful. We know strategies that game designers can put in place to mitigate the negative effects of software evolution. Therefore, we now apply meta-programming techniques to implement these mitigation strategies by leveraging the connection between code and game mechanics. 

We propose a system of solutions that can be generally applied to any game written using PuzzleScript. These tests provide rapid feedback in a live programming context. Instead of having to run a long test suite, the tool responds live to changes and provides the designer feedback live. This system of solutions is implemented as a tool, called ScriptButler. The tool is more efficient than a player because it does not require a full game. With only a few lines of code, it can already start providing feedback on the game.

We note that this does not make human testers redundant. Human testers are much better are detecting defects with the visual aspect of the game. Humans are more creative,  they can break the game in more interesting ways. Although the extensible nature of our tool means that if new, general ways of breaking game are discovered, they can be incorporated. Finally, humans actually experience the game as a whole and can provide feedback on that experience. Our tool simply tests a few interconnected components. ScriptButler empowers game designers by reducing their dependence on playtesting. The playtesters can then focus on the testing more complex problems that actually relate to the way humans experience a game. 

\section{Threats To Validity}
\label{sec:threats}
\paragraph{Game Designer Involvement}
No game designers were actually involved in the process of creating our tools, but we still want our tools to be useful to game designers. This is a threat to validity in that we were not specifically told by game designers that the issues we identified are indeed issues. Instead, we relied on data we gathered from personal experience and from observing the evolution of PuzzleScript games. We mitigate this as much as possible by grounding our research using a case study. The case study mimics scenarios of realistic software evolution. We apply ScriptButler to those scenarios to see to what extent the feedback provided evaluates the impact of those changes. Since the case is simple, we can also validate the errors raised by ScriptButler manually.

% method

\paragraph{Reverse Engineering}
Reverse engineering is not a perfect method for extracting a design from source code. This is illustrated in our case study where we misinterpreted the observed behavior causing an error that does not actually happen in the JavaScript implementation. This error happens because we did not specifically test for this kind of behavior when reverse-engineering the engine. We partially mitigated this issue by conducting extensive testing on our engine using published games, which we first checked against the original PuzzleScript engine. However, this method is not completely reliable since we also try to test for new types of errors within our implementation.

\paragraph{Incomplete Engine}
Our implementation of PuzzleScript does not cover all the features of the original implementation. As such, there is a possibility that the features that remain unimplemented cannot be implemented using Rascal. However, we strongly believe that this will not occur as we have successfully implemented rules, the most complex PuzzleScript features. 

% and as such believe that it will also be possible to implement other functionality such as rigid bodies and loops.

\paragraph{Academic Prototype}
ScriptButler is an academic prototype. Additional work is required to make the tool suitable for professional work. However, ScriptButler has sufficient feature coverage to be used within the context of our project. We needed an extendable implementation of PuzzleScript which allowed us to run PuzzleScript games and perform dynamic analysis on them and the tool met that need. If the tool was required to meet the needs of professional game developers, work would have to go into optimizing the engine and covering the remaining corner cases. Additionally, the formal grammar we have created is still unable to parse more complex combinations of lexicals, especially when combined with code that is heavily commented. Future work should look into making it possible to parse comments as they may contain interesting possible metrics to judge code and gameplay quality\cite{TAN2015493}.


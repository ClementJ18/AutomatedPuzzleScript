\subsection{Compiler}
Game languages require compilers to optimize their performance. Compilers transform the code into lower-level machine code. For PuzzleScript, this means converting the DSL into Rascal data structure. ScriptButler already does part of this task during the parsing, the compilation process completes it. The compiler transforms the existing data structures further, this increases performance but reduces readability. The new data structures do not resemble the original code at all. It may even be impossible to generate code back from the compiler because the structure is lost. For instance, the compiler resolves all references, replacing them with the list of object referenced. As a result, the original reference is lost. Our compiler is separated into three phases, representing the three sections that need to be compiled. Sounds are not compiled because they are not implemented. Objects are not compiled because our tool only requires their name. As such, we only compile three sections: Win Conditions, Levels, and Rules. All three are explained in detail below.

\subsubsection{Levels}
In PuzzleScript, game designers create 2D representations of their levels with the use of legend and aggregates. However, levels are actually 3-dimension when factoring in the collision layer. As a result, we compile the levels into 3D arrays. The design decision we have to take is in which order to arrange the array and their relation to the coordinates. Each object on a level has an XYZ coordinate. X represents the row, Y the column, and Z the layer they are present on. Each of these triples is unique. We have two choices, we can either store the levels as a XYZ array or as a ZXY array. A XYZ array is an array where the first dimension represents the X coordinate, the second the Y coordinate and the third the Z coordinate. An XYZ array has a similar concept but in a different order. The order matters depending on how we match the rules. We decided on an ZXY\dd design, where the first dimension is the layer. This makes it easier to build the level layer by layer. Using the other method would allow us to build levels one row at a time but makes it harder to check using our rule system.

Levels are designed to be able to function standalone. Each level stores all the data required to function. Only the rules and win conditions are required to run a level. However, to render a level, the engine also provides access to the object sprite even if it does not use it currently. During compilation, references and aggregation are resolved and a background layer is generated for rendering purposes. The data structure makes it possible for the engine to store the contents of a level in a simple structure.

\subsubsection{Win Conditions}
Compiling win conditions is a simple process. We resolve references and store them in data structure that makes it easier to check against the 3D structures representing our levels. Levels are also compiled in a way that allows the engine to check for their contents efficiently. Because we store the contents of a level in a simple structure, we can easily check the prerequisites of certain win conditions. For instance, if a level does not contain any instance of ObjectX then the condition \texttt{'No ObjectX on ObjectY'}. This helps performance as it does not require us to go through the level pixel by pixel. The engine can simply access the list of objects contained in the level and make assumptions based on that. However, if there is an instance of ObjectX present, we then have to conduct the more expensive check.

\subsubsection{Rules}
Rules are the most complex part of PuzzleScript and we aim to simplify them in ScriptButler. This goal means that we aim to make the system simpler to extend and we make the rules simpler to debug from a game designer perspective. To this end, we leverage a feature of PuzzleScript called Abstract Patterns\footnote{\url{https://tutor.rascal-mpl.org/Rascal/Patterns/Abstract/Abstract.html}}\dd. Abstract Patterns can be used to match the code with patterns. We use it to match data structure representing a level. However, Rascal does not have native support for creating these Abstract Patterns. The patterns are intended to be used statically to match specific code patterns. We need to be able to generate Abstract Patterns from the Rules written by game designers. We achieve this by generating Abstract Patterns as code to be interpreted by our engine\dd. This design decision incurs a performance loss, as the Rascal code is not compiled but interpreted.

Figure \ref{fig:Rule Compilation Code} shows a simple Rule written in PuzzleScript. This rule allows the player to push a crate. Using our compiler, we transform the rule into a Rascal Abstract pattern that can be seen in \ref{fig:Rule Compilation Pattern}. The pattern represents a 3D array similarly to the compiled form of a Level. In this instance, the game has three layers, as such the first dimension of the array is three elements long. However, the objects referenced in the original layer only exist on the third layer. As such, the first two layers (lines 2 and 3) match for anything and then pass the match to the replacement, changing nothing. Line 4 and 5 unpack the second and third dimension of the level, ensuring that whatever comes after and before the match is passed to the replacements pattern. Finally, lines 6 and 7 handle the actual matching. Line 7 matches a player moving in a direction and Line 8 matches a crate standing still. A similar pattern is used to replace the crate with a moving crate.

\begin{figure}[!t]
    \begin{subfigure}{1\textwidth}
        \begin{lstlisting}[language=rascal]    
        1 # [ 
        2 #     [ *layer0 ], 
        3 #     [ *layer1 ], 
        4 #     [ *prefix_lines2, 
        5 #         [ *prefix_objects2, 
        6 #             Object player0 : moving_object(str name0_0_2 : /player/, int id0_0_2, str direction0_0_2 : relative_right, Coords coords0_0_2 : <xcoord0, ycoord0, zcoord0_0_2>), 
        7 #             Object crate0 : object(str name1_0_2 : /crate/, int id1_0_2, Coords coords1_0_2 : <xcoord1, ycoord1, zcoord1_0_2>), 
        8 #         *suffix_objects2 ], 
        9 #     *suffix_lines2 ]
        10# ]
        \end{lstlisting}
        \vspace*{-8pt}
        \caption{An abstract pattern of the rule in Figure \ref{fig:Rule Compilation Code}}
        \label{fig:Rule Compilation Pattern}
    \end{subfigure}
    
    \begin{subfigure}{1\textwidth}
        \begin{lstlisting}[language=PuzzleScript]
        [ >  Player | Crate ] -> [  >  Player | > Crate  ]
        \end{lstlisting}
        \vspace*{-8pt}
        \caption{A simple PuzzleScript Rule allowing pushing Crates}
        \label{fig:Rule Compilation Code}
        \vspace*{-8pt}
    \end{subfigure}
    
    \caption{Compiling rules}
    \label{fig:compiling rules}
\end{figure}

The compiler requires multiple passes to compile a rule into a list of abstract patterns. However, the end result is more readable from the perspective of a tool designer. Our aim with this design decision is to make it easier to extend PuzzleScript with additional features in rules. Validating Abstract Patterns is easier than validating functions like the original implementation generates. 

The result of the compiler is a list of data structures for Win Conditions, Rules and Levels which can be run by the engine. The data structures we compile are easier to extend than the original implementation. This decision does not change anything for the game designer besides a performance impact caused by the new design focus on extensibility and maintainability. 
